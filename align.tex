\documentclass{scrartcl}
\usepackage{units}
\title{Preliminary Installation Manual for MEMI system}
\author{Martin Kielhorn}
\begin{document}
\section{Introduction}
This document contains instructions for the setup and the alignment of
the MEMI system. It can also serve as a basic introduction to the
system.

The micro-mirror enhanced micro imaging (MEMI) project was funded by the
European Union Framework Programme 7. The system 

\section{List of steps for the Alignment}
\newcommand{\q}[1]{\subsection{#1}} 
\newcommand{\qq}[1]{\subsubsection{#1}} 

\q{Verification of MEMI internal alignment} Place MEMI system onto the
table but don't install microscope just yet.  

\qq{Verify homogenity of MMA illumination} \unit[250]{mm} in front of
the first lens of the MEMI tubelens. Connect the LCoS to a
computer. The laser doesn't have to be triggered but if not triggered,
its intensity should be reduced by a factor of 30. Display a pattern
(e.g. Checkerboard pattern) on MMA, fully open the Fourier apertures
and move a screen or camera into the image of the MMA.

Rotate the microlenses to make the laser illumination spatially
incoherent. It should only be necessary to adjust the coupling of
laser into fibre bundle. The coupling of the fibre bundle into the
MEMI system should already be aligned and shouldn't have moved during
transport.

\qq{Find necessary magnifications} Calculate the diameters of the back
focal plane for the objectives that will be used
($f_\textrm{obj}=\unit[164.5]{mm}/\textrm{MAG}$, $d=2 f_\textrm{obj}
\textrm{NA}$). Make sure you can reach this setting by turning
(zooming) the MEMI tubelens and checking the size of the MMA images.

\qq{Align MMA angle} Whenever the MMA chip is replaced its angle must
be corrected. Put a white card directly in front of the LCoS. Check
that the images of the two Fourier apertures on this screen form
concentric disks. If they don't, loosen the small screws behind the
MMA close illumination aperture to a small hole, open Fourier aperture
and adjust MMA angle, so that the diffraction pattern is symmetric
inside of a rectangle. Then open illumination aperture until the disks
touch and close the Fourier aperture until you see this stop.  Bring
outside stop and bright zero order onto the same center.  When
finished with the alignment screw in the small screws lightly for
fixation.

\qq{LCoS adjustment} The angle of the LCoS should be aligned. The MMA
image should be centered the line through the tubelens. If this is
obviously wrong put a big screen into the MMA image and tilt the LCoS
until the MMA image disappears on left, right, bottom and top. These
four points define a circle. The LCoS angle is correct, when the MMA
image is in the center of this circle.

\q{Align microscope} 

\qq{Modify microscope} Remove illumination optics of the microscopes
back port. You may have to disconnect a shutter cable, in that
carefully label it and indicate where it was connected.  The
microscope should not stand on rubber feet. It should be possible to
screw it into the table at an arbitrary position.


\qq{Verify optical axis of microscope} Focus on some sample with
brightfield and a low magnification ($10\times$) objective. Adjust for
Koehler illumination and close field stop to a small hole. Choose a
filter set where light exits the back port and use a card and the
z-position of the Condensor to follow the beam path. Verify, that it
is parallel to the table. If not, use metal blades to align it.



\end{document}